\documentclass[10pt,letterpaper]{amspset}
\usepackage{fullpage}
\usepackage{pgf,tikz}
\usetikzlibrary{arrows}
\usepackage{enumitem}
\usepackage{lmodern}


\vfuzz2pt % Don't report over-full v-boxes if over-edge is small
\hfuzz2pt % Don't report over-full h-boxes if over-edge is small
% THEOREMS -------------------------------------------------------
\newtheorem{thm}{Theorem}[section]
\newtheorem{cor}[thm]{Corollary}
\newtheorem{lem}[thm]{Lemma}
\newtheorem{prop}[thm]{Proposition}
\theoremstyle{definition}
\newtheorem{defn}[thm]{Definition}
\theoremstyle{remark}
\newtheorem{rem}[thm]{Remark}
\newtheorem{case}{Case}
\newtheorem{subcase}{Case}
\numberwithin{subcase}{case}
\numberwithin{equation}{section}
\renewcommand{\qedsymbol}{\ensuremath{\scriptscriptstyle\blacksquare}}
% MATH -----------------------------------------------------------
\newcommand{\norm}[1]{\left\Vert#1\right\Vert}
\newcommand{\powerset}[1]{\mathcal{P}\left(#1\right)}
\newcommand{\abs}[1]{\left\vert#1\right\vert}
\newcommand{\set}[1]{\left\{#1\right\}}
\newcommand{\Real}{\mathbb R}
\newcommand{\eps}{\varepsilon}
\newcommand{\To}{\longrightarrow}
\newcommand{\BX}{\mathbf{B}(X)}
\newcommand{\A}{\mathcal{A}}
\newcommand{\R}{\mathbb{R}}
\newcommand{\Q}{\mathbb{Q}}
\newcommand{\Z}{\mathbb{Z}}
\newcommand{\N}{\mathbb{N}}

\makeatletter
\renewcommand\@seccntformat[1]{}
\makeatother

\name{Zach Seymour}
\class{CS 575}
\assignment{Theory Assignment \#1}
\duedate{September 10, 2013}



\begin{document}

\problemlist{}

\section{Part I}

\begin{problem}[1]
$\sum^2_{i=0} 3 = 9$
\end{problem}

\begin{problem}[2]
$\prod\limits_{i=0}^{2} = 27$
\end{problem}

\begin{problem}[3]
$\log_3 1 = 1$
\end{problem}

\begin{problem}[4]
$\lg\lg 32 = \lg5$
\end{problem}

\begin{problem}[5]
$\lg^2 8 = (lg 8)^2 = 9$
\end{problem}

\begin{problem}[6]
$\lim\limits_{n\to\infty} (\frac{1}{2})^n = 0$
\end{problem}

\begin{problem}[7]
$\lim\limits_{n\to\infty} 2^n = \infty$
\end{problem}

\begin{problem}[8]
$\frac{d}{dx}\ln x = \frac{1}{x}$
\end{problem}

\begin{problem}[9]
$\frac{d}{dx}e^x = e^x$
\end{problem}

\begin{problem}[10]
$4^{\log_3 n} = n^x$. What is x?
\end{problem}

\begin{solution}
\begin{align*}
x \ln n  &= \ln \left(4^{\log_3 n}\right) \\
x &= \frac{\ln \left(4^{\log_3 n}\right)}{\ln n} 
\end{align*}
\end{solution}

\begin{problem}[11]
Prove by induction that $1+3+5+7+\dotsb+(2(n-1)+1) = n^2$. (Your proof should include the following: proof of base case, the induction hypothesis, what you are trying to prove, the proof. See document in Blackboard on induction proofs)
\end{problem}

\begin{solution}
\begin{prop}
$\sum\limits_{i=0}^{n-1} (2i+1) = n^2$.
\end{prop}

\begin{proof}
For $n=1$, we have $\sum_{i=0}^{0} (2i+1) = 2\cdot0 + 1 = 1$.  So the base case is satisfied.

Now, $\forall k > 1$ we assume $\sum\limits_{i=0}^{k-1} (2i+1) = k^2$.  Therefore, we aim to show $\sum\limits_{i=0}^{k} (2i+1) = (k+1)^2$.  We have 

\begin{align*}
\sum\limits_{i=0}^{k} (2i+1) &= k^2 + \sum\limits_{i=k}^{k} (2i+1) \\
{} &= k^2 + 2k+1 \\
{} &= (k+1)^2
\end{align*}
\end{proof}
\end{solution}

\begin{problem}[12]
Use strong induction and the identity $F_i = F_{i-1} + F_{i-2}$ to prove that the $i$th Fibonacci number satisfies
\[
F_i = \frac{\phi^i - \hat{\phi}^i}{\sqrt{5}}
\]
where $\phi = \frac{1+\sqrt{5}}{2}$ and $\hat{\phi} = \frac{1-\sqrt{5}}{2}$. (Hint: $\phi ^2 = \frac{3+\sqrt{5}}{2}$ and $\hat{\phi}^2 = \frac{3-\sqrt{5}}{2}$)
\end{problem}

\begin{solution}
\begin{proof}
For i=1, we have
\begin{align*}
F_1 &= \frac{\left(\frac{1+\sqrt{5}}{2}\right)^1 - \left(\frac{1-\sqrt{5}}{2}\right)^1}{\sqrt{5}} \\
{} &= \frac{\sqrt{5}}{\sqrt{5}} \\
{} &= 1
\end{align*}

And, for i=1, we have
\begin{align*}
F_2 &= \frac{\left(\frac{1+\sqrt{5}}{2}\right)^2 - \left(\frac{1-\sqrt{5}}{2}\right)^2}{\sqrt{5}} \\
{} &= \frac{\sqrt{5}}{\sqrt{5}} \\
{} &= 1
\end{align*}

Now, let $i \geq 2$.  Assume $\forall k, 1 \leq k \leq i, F_k = \frac{\phi^k - \hat{\phi}^k}{\sqrt{5}}$.  We will show $F_{k+1} = \frac{\phi^{k+1} - \hat{\phi}^{k+1}}{\sqrt{5}}$.

We know $F_{k+1} = F_k + F_{k-1}$.  So, we have
\begin{align*}
F_{k+1} &= F_k + F_{k-1} \\
{} &= \frac{\phi^k - \hat{\phi}^k}{\sqrt{5}} + \frac{\phi^{k-1} - \hat{\phi}^{k-1}}{\sqrt{5}}\\
{} &= \frac{\left(\phi^k - \hat{\phi}^k\right) + \left(\phi^{k-1} - \hat{\phi}^{k-1}\right)}{\sqrt{5}} \\
{} &= \frac{\phi^k + \phi^{k-1} - \hat{\phi}^k - \hat{\phi}^{k-1}}{\sqrt{5}} \\
{} &= \frac{\phi^{k-1}(\phi + 1) - \hat{\phi}^{k-1}(\hat{\phi}+1)}{\sqrt{5}} \\
{} &= \frac{\phi^{k-1}\phi^2 - \hat{\phi}^{k-1}\hat{\phi}^2}{\sqrt{5}} \\
{} &= \frac{\phi^{k+1} - \hat{\phi}^{k+1}}{\sqrt{5}}
\end{align*}

\end{proof}
\end{solution}

\begin{problem}[13]
$\sum_{i=1}^{n} n-1 = n \cdot (n-1)$
\end{problem}

\begin{problem}[14]
$\sum_{i=1}^{n} 3^i = 3^1 + 3^2 + \dotsb + 3^n = \frac{1-3^{n+1}}{-2}$
\end{problem}

\begin{problem}[15]
In how many different orders can $x$ students sit around a round table?  Two orders are identical if one can be rotated to form the other.
\end{problem}

\begin{solution}
There are $n!$ different arrangements, each with $n$ rotations, so $(n-1)!$ different orders.
\end{solution}

\section{Part II}

\begin{problem}[1]
\texttt{CountMe} has count $n^4$.
\end{problem}

\begin{problem}[2]
\texttt{CountMe} has count $\left\lceil\frac{n}{2}\right\rceil$ for $n$ even and $\left\lfloor\frac{n}{2}\right\rfloor$ for $n$ odd.
\end{problem}

\begin{problem}[3]
\texttt{CountMe} has count $\left\lceil\frac{n}{2}\right\rceil$ for $n$ odd and $\left\lfloor\frac{n}{2}\right\rfloor$ for $n$ even.
\end{problem}

\begin{problem}[4]
Given the pseudo code below for bubble sort...Let $length[A] = n$. Derive the count of the number of times that the comparison $(A[j] < A[j - 1])$ is executed by the algorithm. Show how you derived your answer. 
\end{problem}

\begin{solution}
If $length[A] = n$, the outer loop executes $n-1$ times. The inner loop executes $\sum_{i=2}^{n} i = \frac{n^2 + n - 2}{2}$ times.  So, the overall instruction count is $(n-1)\left(\frac{n^2 + n - 2}{2}\right)$.
\end{solution}

\begin{problem}[5]
Suppose we are comparing implementations of bubble sort and merge sort on the same machine.  For inputs of size $n$, bubble sort runs in $15n^2$ steps, while merge sort runs in $150n\lg n$ steps.  For which values of $n$ does bubble sort beat merge sort? (Hint: use Excel write a program, or try different values).  What is the number of steps done by each for $n=1024$? 
\end{problem}

\begin{solution}
Bubble sort beats merge sort for all values $n, 1 < n < 59$, using the following Mathematica code.
\begin{verbatim}
N[Solve[15 n^2 == 150*n*Log[2, n], n]]
\end{verbatim}
For $n=1024$, bubble sort takes 15,728,640 steps, while merge sort takes 1,536,000.
\end{solution}

\begin{problem}[6]
What is the smallest value of $n$ such that an algorithm whose running time is 
$2^{20} n$ runs faster than an algorithm whose running time is $2^n$ on the same machine? 
\end{problem}

\begin{solution}
Once $n$ is larger than 24, $2^n$ exceeds $2^{20} n$.  So the smallest such $n$ is 25.
\end{solution}

\section{Extra Credit}
\begin{problem}
Prove by induction that the following program computes $n!$ correctly for all integers $n \geq 1$.
\end{problem}

\begin{solution}
\begin{proof}
For initialization, we need to show the program is correct when $n=1$.  Indeed, given the condition \verb!if(n==1) return 1!, we correctly have $1! = 1$.

For maintenance, we observe that for each call where $n > 1$, the program returns $n$ multiplied by a call to the next lowest factorial, which is correct with regards to the formula for the factorial function, $n! = \prod\limits_{i=1}^{n} i$.

Finally, for termination, we notice that, when repeatedly subtracting one from $n$, where $n > 1$, we will eventually have $n = 1$, whereby the program will terminate.
\end{proof}
\end{solution}
\end{document}
