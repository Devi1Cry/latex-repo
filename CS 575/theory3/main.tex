\documentclass[12pt,letterpaper]{hmcpset}
\usepackage[margin=1in]{geometry}
\usepackage{graphicx}
\usepackage{algpseudocode}
\usepackage{zach}

% info for header block in upper right hand corner
\name{Zachary Seymour}
\class{CS 575}
\assignment{Theory Assignment 3}
\duedate{October 17, 2013}

\begin{document}

\problemlist{}

\begin{problem}[1]
Given an undirected graph $G$, modify the code of DFS to check if $G$ contains cycles.  If it does, print the first one discovered and return ``yes'', otherwise return ``no''.
\end{problem}

\begin{solution}
Our original pseudocode for depth-first search consists of two functions:
\begin{algorithmic}
\Function{DFS}{G, color, d, f, parent}
\ForAll{vertex u}
	\State color[u]=white
	\State parent[u]=-1
	\State time=0
\EndFor
\ForAll{vertex u}
\If{color[u]==white}
\State DFS-Visit(u)
\EndIf
\EndFor
\EndFunction
\end{algorithmic}
and

\begin{algorithmic}
\Function{DFS-Visit}{u}
\State color[u]=gray
\State time=time+1
\State d[u]=time
\ForAll{v $\in$ adj[u]}
\If{color[v]==white}
\State parent[v]=u
\State DFS-Visit(v)
\EndIf
\EndFor
\State color[u]=black
\State time=time+1
\State f[u]=time
\EndFunction
\end{algorithmic}

We modify the \code{DFS-Visit} function to check if any each $v$ adjacent to vertex $u$ have been marked gray, which means this is the second time we are visiting it while checking the same node, so a cycle exists.  Otherwise, we recurse as normal.
So, we now have:
\begin{algorithmic}
\Function{HasCycle?}{G, color, d, f, parent}
\ForAll{vertex u}
	\State color[u]=white
	\State parent[u]=-1
	\State time=0
\EndFor
\ForAll{vertex u}
\If{color[u]==white}
\If{DFS-Visit(u)}
\State print cycle
\State \Return ``yes''
\EndIf
\EndIf
\EndFor
\State \Return ``no''
\EndFunction
\end{algorithmic}
and

\begin{algorithmic}
\Function{DFS-Visit}{u}
\State color[u]=gray
\State time=time+1
\State d[u]=time
\ForAll{v $\in$ adj[u]}
\If{color[v]==gray}
\State \Return true
\ElsIf{color[v]==white}
\State parent[v]=u
\If{DFS-Visit(v)}
\State \Return true
\EndIf
\EndIf
\EndFor
\State color[u]=black
\State time=time+1
\State f[u]=time
\State \Return false
\EndFunction
\end{algorithmic}
\end{solution}

\begin{problem}[2]
For each node $u$ in an undirected graph, let \verb|sDegree(u)| be the sum of the degrees of the neighbors of $u$.  Give pseudocode for an $\BigO{E+V}$ algorithm that outputs for each node $u$ its \code{sDegree}.
\end{problem}

\begin{solution}
\begin{algorithmic}
\Function{sDegree}{u}
\State sDegree=0
\ForAll{v $\in$ adj[u]}
\ForAll{t $\in$ adj[v], s.t. t$\neq$u}
\State sDegree=sDegree+1
\EndFor
\EndFor
\EndFunction
\end{algorithmic}
\end{solution}

\begin{problem}[3]
The reverse of a directed graph $G=(V,E)$ is another directed graph $G^R=(V,E^R)$ on the same vertex set, but with all edges reversed.  So $E^R = \set{(v,u)|(u,v) \in E}$.  Give an $\BigO{E+V}$ algorithm for computing the reverse of a graph in adjacency list format.
\end{problem}

\begin{solution}
To reverse the adjacency list, rather than transposing the adjacency matrix, we cycle through all nodes $u$ and for each edge to $v$, add $u$ to the beginning of the adjacency list for $v$.
\begin{algorithmic}
\ForAll{u $\in$ V}
\ForAll{v $\in$ adj[u]}
\State adj[v]$\gets$u
\EndFor
\EndFor
\end{algorithmic}
\end{solution}

\begin{problem}[4]
Assume the graph of the Web where a link $y$ stored in page $x$ is represented by a directed edge from $x$ to $y$.  Write pseudocode for printing the address of all the pages at distance $d$ from $p$. (The address can be derived from \verb|x.address|).
\end{problem}

\begin{solution}
I'm going to use a modification of \code{DFS-Visit}, keeping track of how far away from $x$ we've traveled, printing the address once we have gone $d$ steps down.  I'll also assume that each node is at distance zero from itself.
\begin{algorithmic}
\Function{PagesAtDistance}{d, x}
\If{d==0}
\State print x.address
\Else
\ForAll{y $\in$ adj[x]}
\State PagesAtDistance(d-1,y)
\EndFor
\EndIf
\EndFunction
\end{algorithmic}
\end{solution}

% Add pairs of problems and solutions as needed

\end{document}
