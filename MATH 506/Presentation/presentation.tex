\documentclass[12pt,letterpaper,twoside]{hmcpset}
\usepackage[margin=1in]{geometry}
\usepackage{graphicx}
\usepackage{commons}
\usepackage{accents}

% info for header block in upper right hand corner
\name{Zachary Seymour}
\class{MATH 506}
\assignment{Presentation Notes}
\duedate{May 6, 2014}

\begin{document}

\begin{problem}
  The space $R^\infty$ is complete.
\end{problem}

\begin{solution}
  \begin{proof}
    Let $\set{x^n}$ be a Cauchy sequence in $R^\infty$, where $x^n = (x_1^n, x_2^n, \dotsc)$.
    Our goal is to find an $x = (x_1, x_2, \dotsc) \in R^\infty$ such that $x^n \to x$.
    Let $\eps > 0$.
    Because $\set{x^n}$ is Cauchy, we have $\exists N > 0$ such that $d(x^n,x^m) < \eps, \forall n,m > N$.
    Therefore, for all $i$ and all $n,m > N$, we have $\Abs{x^n_i - x^m_i} < \eps$.
    Furthermore, this means, for each $i$ we have $(x^1_i, x^2_i, \dotsc)$ a Cauchy sequence in $\R$.
    Since $\R$ is complete, we have a limit $x_i$: $x_i = \lim_{n\to\infty}x^n_i$.
    
    Now, choose $\eps > 0$ and replace $\eps$ above with $\frac{\eps}{2}$.
    Again, we can find an $N > 0$ such that $\Abs{x_k^n - x_k^m} < \frac{\eps}{2}$, for all $k$ and $n,m > N$.
    Now take the limit as $m \to \infty$, we have $\Abs{x_k^n - x_k} < \frac{\eps}{2}$.
    If we take supremum, we have $\sup_{k=1}^\infty\Abs{x_k^n - x_k} \leq \frac{\eps}{2}$.
    This implies $d(x^n, x) \leq \frac{\eps}{2} < \eps$ and thus $x^n$ converges to $x$.
    
  \end{proof}
\end{solution}

\begin{problem}
  If $P_n \rightharpoonup P$ in $(M, \mathcal{B}_M)$ and $P_n(M_0) = P(M_0) = 1$, then $(P_n)^r \rightharpoonup P^r$ in $(M_0, \mathcal{B}_{M_0})$.
\end{problem}

\begin{solution}
  \begin{lem}
    \label{lem:1}
    For any open set $G \subseteq M$, $\liminf_{n\to\\infty}P_n(G) \geq P(G)$.
  \end{lem}
  \begin{proof}
    Let $G$ be an open set and let $H = G^c$.
    Consider a sequence of functions $f_m(s) = \min(1,m\cdot d(s,G))$.
    Then, each $f_m$ is a bounded continuous function, $0 \leq f_m \leq 1$ and $f_m \to 1$.
    From the weak convergence of $P_n$, we have \[
    \Int{f_m}{P} = \liminf_m \Int{f_m}{P_n} \leq \liminf_n P_n(G).
    \]
    Letting, $m\to\infty$, we get $\liminf_n P_n(G) \geq P(G)$.
  \end{proof}
  
  By Remark 4.2, we have $P(G) = P^r(G \cup M_0)$, and we will define $G_0 \coloneqq G \cup M_0$, the general open set in $M_0$.
  Since we also have $P_n(M_0) = 1$, we can also write $P_n(G)$ as $(P_n)^r(G_0)$.
  Therefore, we have \[
  \liminf_n (P_n)^r(G_0) \geq P^r(G_0).
  \]
  As shown before, if we take the complement above, we also get \[
  \limsup_n (P_n)^r(G_0) \geq P^r(G_0).
  \]
  
  Now, let us take an $A \in \mathcal{B}_{M_0}$ such that $P^r(\partial A) = 0$.
  Also, let $G = \accentset{\circ}{A}, K = \cl{A}$, so we have $G \subseteq A \subseteq K$ and $\partial A = K \setminus G$.
  For every $n$, then, we have $P_n^r(G) \leq P_n^r(A) \leq P_n^r(K)$ so that \[
  \liminf_n P_n^r(G) \leq \liminf_n P_n^r(A) \leq \limsup_n P_n^r(A) \leq \limsup_n P_n^r(K).
  \]
  From our previous result, we can deduce that $P^r(G) \leq \liminf_n P_n^r(A) \leq \limsup_n P_n^r(A) \leq P^r(K)$.
  Therefore, since $P^r(K\setminus G) = 0$, we have $P^r(G) = P^r(A) = P^r(K)$, and thus \[
  \lim_n P_n^r(A) = P^r(A).
  \]
  
  Finally, since we have $\lim_n P_n^r(A) = P^r(A)$ for all $A \in \mathcal{B}_{M_0}$ such that $P^r(\partial A) = 0$, we can conclude that for each $f \in C_b(M_0)$, \[
  \Int{f}{P_n^r,M_0} \to \Int{f}{P^r,M}
  \]
  which is the definition of weak convergence.
\end{solution}
\end{document}
%%% Local Variables: 
%%% mode: latex
%%% TeX-master: t
%%% End: 
