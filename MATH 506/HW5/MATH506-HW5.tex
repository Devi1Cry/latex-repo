
\documentclass[12pt,letterpaper,twoside]{hmcpset}
\usepackage[margin=1in]{geometry}
\usepackage{graphicx}
\usepackage{commons}

% info for header block in upper right hand corner
\name{Zachary Seymour}
\class{MATH 506}
\assignment{Homework 5}
\duedate{April 8, 2014}

\begin{document}
\noindent Prove the following statements:
$\func{f}{\R}{\R}$
\begin{problem}[1]
  Let $H$, $K$, and $L$ be Hilbert spaces, let $R,S \in B(H,K)$ and let $T \in B(K,L)$. Let $\lambda, \mu \in \R$.
  Then
  \begin{enumerate}[label=(\alph*)]
  \item $(\mu R + \lambda S)^* = \mu R^* + \lambda S^*$;
  \item $(TR)^* = R^*T^*$.
  \end{enumerate}
\end{problem}

\begin{solution}
  \begin{proof}
    For (a), we have for all $x \in H$ and $y \in K$
    \begin{align*}
      (x,(\mu R + \lambda S)^* y) &= ((\mu R + \lambda S) x, y) \\
      {} &= (\mu Rx, y) + (\lambda Sx, y) \\
      {} &= (x, \mu R^* y) + (x, \lambda S^* y) \\
      {} &= (x, (\mu R^* + \lambda S^*) y )\\
    \end{align*}
    Thus, for all $y$, we have $(\mu R + \lambda S)^* = \mu R^* + \lambda S^*$.
    
    For (b), we have \[
    (x,(TR)^* y) = ((TR)x, y) = (Rx, T^*y) = (x, R^*T^*y).\
    \]

  \end{proof}
\end{solution}

\begin{problem}[2]
  Let $H$ be a Hilbert space and let $T \in B(H)$. Then
  \begin{enumerate}[label=(\alph*)]
  \item $\Ker(T) = \Ker(T^*T).$
  \item $\cl{\Image(T^*)} = \cl{\Image(T^*T)}$.
  \end{enumerate}
\end{problem}

\begin{solution}
  \begin{proof}
    \label{prf:1}
    For (a), we first show that $\norm{T^*} = \norm{TT^*} = \norm{T}^2$.
    From the Schwarz inequality, we have \[
    \norm{Tx}^2 = (Tx, Tx) = (T^*Tx, x) \leq \norm{T^*Tx}\norm{x} \leq \norm{T^*Tx}\norm{x}^2. 
    \]
    Then, taking the supremum over all $x, \norm{x} = 1$, we have $\norm{T}^2 \leq \norm{T^*T}$.
    Applying some properties of adjoint, we then have \[
    \norm{T}^2 \leq \norm{T^*T} \leq \norm{T^*}\norm{T} = \norm{T}^2.
    \]
    Thus, $\norm{T^*T} = \norm{T}^2$. 
    Replacing $T$ with $T^*$, we have also $\norm{T^{**}T^*} = \norm{TT^*} = \norm{T}^2$, since $T^{**} = T$.
    From this result, we have $T^*T = 0$ iff $T = 0$, thus, the kernels are equal.
  \end{proof}
\end{solution}

\begin{problem}[3]
  Let $H$ be a Hilbert space, and $T \in B(H)$ be a self-adjoint operator.  Then for all $\mu \in V(T)$,
  \[
  \inf \set{\lambda : \lambda \in \sigma(T)} \leq \mu \leq \sup\set{\lambda : \lambda \in \sigma(T)}.
  \]

\end{problem}

\begin{solution}
  \begin{proof}
    \label{prf:5}
    For self-adjoint operators we have $\inf\sigma(T) = \inf\set{(Tx,x) : \norm{x} = 1}$ and $\sup\sigma(T) = \sup\set{(Tx,x) : \norm{x} =1 }$.
    Furthermore, for $T$ bounded and self-adjoint, we can define $\func{v}{H}{\R} : x \mapsto (Tx, x)$.
    Since $V(T) = v(B)$, where $B = \set{x \in H : \norm{x} = 1}$ and $T$ is bounded, we conclude $V(T)$ is bounded.
    Furthermore, since $B$ is connected and $V(T)$ is the continuous of image of $B$, we have $V(T)$ connected and bounded.
    Therefore, we can express $V(T)$ as an interval at least of the form $(m,M)$, where \[
    m = \inf v(x) = \inf\sigma(T) \qquad \text{and} \qquad M = \sup v(x) = \sup\sigma(T).
    \]
    Thus, each $m \leq \mu \leq M$ for each $\mu \in V(T)$.
  \end{proof}
\end{solution}

\begin{problem}[4]
  Let $H$ be a Hilbert space.  If $Q$ is an orthogonal projection in $B(H)$ then $\Image(Q)$ is a closed linear subspace and $Q = P_{\Image(Q)}$.
\end{problem}

\begin{solution}
  \begin{proof}
    \label{prf:2}
    Since $Q$ is an orthogonal projection in $H$, we have $H = \Ker(Q) \oplus \Image(Q)$.
    If we take $x = Qy$ and $z \in \Ker(Q)$, then \[
    (x,z) = (Qy,z) = (y, Qz) = 0
    \]
    so we have $\Image(Q) \perp \Ker(Q)$.
    Thus, $H$ is the orthogonal direct sum of the image and kernel of $Q$, so $Image(Q) = \Ker(Q)^\perp$; therefore, $Q$ is closed.
  \end{proof}
\end{solution}

\begin{problem}[5]
  Let $H$ be a Hilbert space and let $P,Q \in B(H)$ be orthogonal projections.
  \begin{enumerate}[label=(\alph*)]
  \item If $PQ = QP$, then $PQ$ is an orthogonal projection.
  \item $\Image(P)$ is orthogonal to $\Image(Q)$ iff $PQ = 0$.
  \end{enumerate}
\end{problem}

\begin{solution}
  \begin{proof}
    \label{prf:3}
    First, we note that since $P$ and $Q$ are projections, we have $P = P^2$ and $Q = Q^2$.
    Also, we have $P$ and $Q$ are self-adjoint.
    For (a), then, we have $PQ = QP = Q^* P^* = (PQ)^*$; thus, $PQ$ is also self-adjoint.
    We then also have $(PQ)^2 = PQPQ = PPQQ = P^2Q^2 = PQ$, so it follows that PQ is a projection.
  \end{proof}
\end{solution}

\begin{problem}[6]
  Let $H$ be a Hilbert space and let $y,z \in H$.  Define $T \in B(H)$ by $Tx = (x,y)z$. Then $T$ is compact.
\end{problem}

\begin{solution}
  \begin{proof}
    \label{prf:4}
    If we take $T = RS$, where $Rx = (x,y)$ and $Sx = z$, where $S$ is definitely compact (since it is constant), then we have two bounded operators, one of which is compact, so their product must be compact.
  \end{proof}
\end{solution}

\end{document}
%%% Local Variables: 
%%% mode: latex
%%% TeX-master: t
%%% End: 
