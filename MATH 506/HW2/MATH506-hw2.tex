\documentclass[10pt,letterpaper,twoside]{hmcpset}
\usepackage[margin=1in]{geometry}
\usepackage{graphicx}
\usepackage{commons}

% info for header block in upper right hand corner
\name{Zachary Seymour}
\class{MATH 506}
\assignment{Homework 2}
\duedate{February 20, 2014}

\begin{document}
Prove the following statements:

\begin{problem}[1]
 Let $\pa{X,d}$ be a metric space.  Then $\pa{X,d}$ is complete iff every nest of closed balls has a nonempty intersection.
\end{problem}

\begin{solution}
 \begin{proof}
  We will first assume $\pa{X,d}$ is complete.  Let $\set{B_n}$ be such a nest, where $B_n$ is the ball with center at $x_n$ and radius $r_n$.  We have $\lim_{n\to\infty} r_n = 0$; therefore, the sequence $\set{x_n}$ is Cauchy, and its limit lies inside every $B_n$, so their intersection is nonempty.
  
  If we have such a nest as described above with $\set{x_n}$ a Cauchy sequence.
  
  And then I'm not sure where to really go from here.
 \end{proof}

\end{solution}


\begin{problem}[2]
 Let $X$ be a linear space and let $\norm{\cdot}_1$ and $\norm{\cdot}_2$ be two norms on $X$.  Let $d_1$ and $d_2$ be the metrics defined by $d_i(x,y) = \norm{x-y}_i$, $i=1,2$. Suppose that there exists $K > 0$ such that $\norm{x}_1 \leq K\norm{x}_2$ for all $x \in X$.  Let $\set{x_n}$ be a sequence in $X$.
\end{problem}

\begin{problem}[2(a)]
 If $\set{x_n}$ converges to $x$ in the metric space $\pa{X,d_2}$, then $\set{x_n}$ converges to $x$ in the metric space $\pa{X,d_1}$.
\end{problem}

\begin{solution}
  \begin{proof}
   
 Suppose $\set{x_n}$ converges to $x$ in $\pa{X,d_2}$.  Then, $\lim_{n\to\infty} \norm{x_n-x}_2 = 0$.  
 We also have $\lim_{n\to\infty} \norm{x_n - x}_1\linebreak\leq K \cdot 0$, so $\set{x_n}$ converges to $x$ in $\pa{X,d_1}$.

  \end{proof}

\end{solution}


\begin{problem}[2(b)]
 If $\set{x_n}$ is Cauchy in the metric space $\pa{X,d_2}$, $\set{x_n}$ is Cauchy in the metric space $\pa{X,d_1}$.
\end{problem}

\begin{solution}
 \begin{proof}
  Suppose $\set{x_n}$ is Cauchy in w.r.t. $\norm{\cdot}_2$ and let $\eps > 0$.  Then, $\exists N > 0$ such that $\norm{x_n - x_m}_2 \leq K\eps$ when $n,m > N$.  Therefore, $\norm{x_n - x_m}_1 \leq \eps$ whenever $n,m > N$, so $\set{x_n}$ is Cauchy in $\pa{X,d_1}$.
 \end{proof}

\end{solution}


\begin{problem}[3]
 If $X$ is a normed linear space and $S$ is a linear subspace of $X$ then $\overline{S}$ is a linear subspace of $X$.
\end{problem}

\begin{solution}
 \begin{proof}
 For $\overline{S}$ to be a linear subspace, we must show that it is addition and scalar multiplication.
 
 Consider $a,b \in \cl{S}$.  Then we consider three cases.
 \begin{case}
  $a,b \in S$.
 \end{case}
  Then, $a + b \in S$ since $S$ is a linear subspace already, so clearly, $a + b \in \cl{S}.$
  \begin{case}
   $a \in \cl{S} \setminus S$, $b \in S$.
  \end{case}
  By definition of closure, then, $\exists a_0 \neq a \in S$ such that $\norm{a_0 - a} < \eps$.  We have, though, $a_0 + b \neq a + b \in S$ and $\norm{a_0 + b - a - b} = \norm{a_0 - a} < \eps$, so the distance between $a + b \in S$ and $a_0 + b$ is arbitrarily small, so we can conclude $a_0 + b \in \cl{S}$.
  \begin{case}
   $a \in \cl{S} \setminus S$, $b \in \cl{S} \setminus S$
  \end{case}
   This case is essentially the same as the previous case, where the distance between some $a_0 + b_0 \neq a + b \in S$ can be made arbitrarily small as well.
   
   Now, if we have some $x \in \cl{S}$ and some scalar $\alpha$, we consider two more cases:
   
   \begin{case}
    $x \in S$
   \end{case}
   That $\alpha x \in \cl{S}$ follows from the fact that $S$ is also a linear subspace.
   \begin{case}
    $x \in \cl{S} \setminus S$
   \end{case}
    Again, a similar argument can be made to show that the distance between some $\alpha x_0 \neq \alpha x$ can be made arbitrarily small, so $\alpha x \in \cl{S}$.
    
    Therefore, we have shown $\cl{S}$ is a linear subspace of $X$.
 \end{proof}
\end{solution}


\begin{problem}[4]
 Let $\mathcal{P}$ be the linear space of polynomials with real coefficients defined on $\bra{0,1}$.  Then we can define the sup-norm $\norm{p}_\infty = \sup\set{\abs{p(x)} : x \in \bra{0,1}}$ and $L^1$-norm $\norm{p}_1 = \Int{\abs{p(x)}}{x,0,1}$.  Show that these two norms are not equivalent on $\mathcal{P}$.
 
 (Hint: consider $p_n(x) = x^n$.)
\end{problem}

\begin{solution}
\begin{proof}
  Suppose the two norms are equivalent. Then $\exists c_1, c_2$ such that $c_1\norm{p}_1 \leq \norm{p}_\infty \leq c_2\norm{p}_1$. For $p_n(x) = x^n$, we have 
 \begin{align*}
  \norm{p_n}_\infty &= \max_{x\in\bra{0,1}} \abs{x}^n = 1 \\
  \norm{p_n}_1 &= \Int{\abs{x}^n}{x,0,1} = \frac{1}{n+1}.
 \end{align*}
Then, for $c_2 > 0$ for all $n$, we have 
\[\frac{\norm{p_n}_\infty}{\norm{p_n}_1} = n + 1 \leq c_2. \]
This cannot hold for all $n$.  Therefore, the norms are not equivalent.
\end{proof}

\end{solution}


\begin{problem}[5]
 If $\pa{X, \norm{\cdot}_1}$ and $\pa{Y,\norm{\cdot}_2}$ are two normed linear spaces, then \[\norm{\pa{x,y}} = \max\set{\norm{x}_1,\norm{y}_2}, \qquad x\in X, y\in Y\] is a norm on the product space $X \cross Y$.
\end{problem}

\begin{solution}
\begin{proof}
  We must show the norm as defined meets the four criteria:
 \begin{enumerate}
  \item $\norm{x} \geq 0$,
  \item $\norm{x} = 0$ iff $x = 0$,
  \item $\norm{\alpha x} = \abs{\alpha}\!\norm{x}$,
  \item $\norm{x + y} \leq \norm{x} + \norm{y}$.
 \end{enumerate}

 That (1) holds follows from the fact that both $\norm{x}_1$ and $\norm{y}_2$ are also nonnegative, so their max must be as well.  Next, $\max\set{\norm{x}_1,\norm{y}_2}$ can only equal zero when $x = 0$ and $y=0$, so the second property holds.  If, we treat scalar multiplication and addition component-wise, the other properties hold as well.
\end{proof}

\end{solution}


\begin{problem}[6]
 The linear space $L^1\pa{M, \mu}$ is complete in its metric $d(f,g) = \Int{\abs{f-g}}{\mu}$. You can use the following procedure to prove.
 
 \begin{enumerate}[label=(\alph*)]
  \item For any Cauchy sequence $\set{f_n} \subset L^1$, find a subsequence $\set{f_{n_k}}$ such that $\norm{f_{n_{k+1}} - f_{n_k}}_{L^1} \leq 2^{-k}$.
  \item Define \[f(x) = f_{n_1}(x) + \sum^\infty_{k=1}\pa{f_{n_{k+1}}(x) - f_{n_k}(x)},\] then $f \in L^1$, $f_{n_k}(x) \to f(x)$ a.e. and $f_{n_k}(x) \to f(x)$ in $L^1$.
  \item Show that $\norm{f_n - f}_{L^1} \to 0$ as $n\to\infty$.
 \end{enumerate}

\end{problem}

\begin{solution}
 \begin{proof}
  Suppose $\set{f_n}$ is Cauchy in $L^1$ and that we pick out a subsequence $\set{f_{n_k}}$ with $\norm{f_{n_{k+1}} - f_{n_k}}_{L^1} \leq 2^{-k}$.  Then, we define the infinite series $f(x)$ by \[f(x) = f_{n_1}(x) + \sum^\infty_{k=1}\pa{f_{n_{k+1}}(x) - f_{n_k}(x)},\] so we have $f \in L^1$, $f_{n_k}(x) \to f(x)$ a.e. and $f_{n_k}(x) \to f(x)$ in $L^1$.  Since infinite series $f(x)$ is dominated by $\sum 2^{-k} \leq 1$, we can say, by the Dominated Convergence Theorem, that $\norm{f_n - f}_{L^1} \to 0$.
 \end{proof}

\end{solution}

\end{document}
