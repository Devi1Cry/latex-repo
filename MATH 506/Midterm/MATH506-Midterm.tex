\documentclass[12pt,letterpaper,twoside]{hmcpset}
\usepackage[margin=1in]{geometry}
\usepackage{graphicx}
\usepackage{commons}

% info for header block in upper right hand corner
\name{Zachary Seymour}
\class{MATH 506}
\assignment{Midterm: Exam 1}
\duedate{March 11, 2014}

\begin{document}
\noindent Prove the following statements.  You need to provide a complete proof of each problem in order to get the full credit.

\begin{problem}[1][10]
 Let $\pa{X,\norm{\cdot}_1}$ and $\pa{Y,\norm{\cdot}_2}$ be two Banach spaces. If $Z = X \times Y$ (Cartesian product) with the norm given by $\norm{(x,y)} = \norm{x}_1 + \norm{y}_2$, then $\pa{Z,\norm{\cdot}}$ is a Banach space.
\end{problem}

\begin{solution}
 \begin{proof}
  Let $\set{(x_n,y_n)}$ in $Z$ with $\set{x_n}$ Cauchy in $X$ and $\set{y_n}$ Cauchy in $Y$.  Since $X$ and $Y$ are Banach spaces, there exist points $x \in X$, $y \in Y$ such that $\lim_{i\to\infty} \norm{x_i - x}_1 = 0$ and $\lim_{i\to\infty} \norm{y_i - y}_2 = 0$.
  
  Consider $\pa{x,y} \in Z$. We have \[\lim_{i\to\infty} \norm{(x_i, y_i) - (x,y)} = \lim_{i\to\infty} \norm{(x_i - x, y_i - y)} = \lim_{i\to\infty}\norm{x_i - x}_1 + \lim_{i\to\infty} \norm{y_i - y}_2 = 0.\]
  
  Therefore, the Cauchy sequence $\set{(x_n,y_n)}$ has a limit in $Z$, so $Z$ is a Banach space.
 \end{proof}
\end{solution}


\begin{problem}[2][10]
 Let $\pa{X, \mathcal{M}, \mu}$ be a $\sigma$-finite measure space.  Then for $1 \leq p < \infty$, $L^p\pa{X,\mu}$ is a Banach space with the norm $\norm{f}_p = \pa{\Int{\Abs{f}^p}{\mu}}^{1/p}$.
\end{problem}

\begin{solution}
 \begin{proof}
  We must show that $L^p\pa{X,\mu}$ is complete in the given norm.
  
  First, let $\set{f_n}$ be Cauchy in $L^p$ with respect to $\norm{\cdot}_p$, 
with $\Sum{\norm{f_n}_p}{n,1,\infty} = M < \infty$. Define $g_n = 
\Sum{f_k}{k,1,n}$. By the Minkowski inequality, then, we have 
\[\norm{g_n}_p \leq \Sum{\norm{f_k}_p}{k,1,n} \leq M.\] By the monotone convergence theorem, $m = \Sum{\Abs{f_k}}{k,1,\infty} \in L^p$ and bounded by $B$.
It follows, then, that $\Sum{f_k}{k,1,\infty}$ converges a.e. to some $f$, that $\Sum{f_k}{j,1,n}$ is dominated by $h$, and that it converges pointwise to $\Sum{f_k}{k,1,\infty}$.
Therefore, by the dominated convergence theorem, we conclude that our sequence $\set{f_n}$ converges to the same $f \in L^p$, and thus $L^p$ is a Banach space. 
 \end{proof}
\end{solution}


\begin{problem}[3][20]
 Let $\set{H_n}_{n=1}^\infty$ be a sequence of Hilbert spaces and let $H = \set{\set{x_n} : x_n \in H_n, \sum\norm{x_n}^2 < \infty}$. For $\set{x_n},\set{y_n} \in H$, $a,b \in \R$, define $a\set{x_n} + b\set{y_n} = \set{ax_n + by_n}$ and $\pa{\set{x_n},\set{y_n}} = \sum\pa{x_n,y_n}$. Then $H$ is a Hilbert space.
\end{problem}

\begin{solution}
 \begin{proof}
  First, we know the given function is an inner product because it relies on 
the sum of inner products from the given sequence of Hilbert spaces.  We need 
just to show the space is complete.
  
  The induced norm on this space is then $\norm{\set{x_n}} = \pa{\sum\pa{x_n,x_n}}^{1/2} = \pa{\sum\norm{x_n}^2}^{1/2}$, which is finite, by definition. Let $\set{x_n}$ be a Cauchy sequence in $H$, where each $\set{x_m} = \pa{x_1^m, x_2^m, \dotsc}$.
 \end{proof}

\end{solution}


\begin{problem}[4][10]
 Let $X$ be an inner product space and let $A \subset X$. then $A^\perp = \cl{A}^\perp$.
\end{problem}

\begin{solution}
 \begin{proof}
  This follows from the continuity of inner product. If for $x \in X$ $(x,a) = 0$ for all $a \in A$, then $x$ will also be orthogonal to all of the limit points of A.
 \end{proof}

\end{solution}


\begin{problem}[5][10]
 Suppose that $H$ is a separable Hilbert space and $Y \subset H$ is a closed 
linear subspace. Then there is an orthonormal basis for $H$ consisting only of 
elements of $Y$ and $Y^\perp$.
\end{problem}

\begin{solution}
 Since $Y$ is a closed linear subspace of $H$, we know that $H$ can be 
described by $H = Y \oplus Y^\perp$ (this is comes from the direct sum theorm). 
That is, every $h \in H$ can be written as $h = y + y'$ for some $y \in Y$ and 
$y' \in Y^\perp$.  We can then find an orthnormal basis $\set{e_n}$ for $Y$ and 
$\set{e_m}$ for $Y^\perp$ and we have $\set{e_n} \bigcup \set{e_m}$ and 
orthonormal basis form $H$.
\end{solution}



\begin{problem}[6][15]
 Let $Y$ be a closed linear subspace of a Hilbert space $H$. If $Y \neq H$, then $Y^\perp \neq \set{0}$. Is this always true if $Y$ is not closed?
\end{problem}

\begin{solution}
 \begin{proof}
  Since $Y$ is a closed linear subspace of $H$, every $H$ can be expressed as 
the sum of an element of $Y$ and an element of $Y^\perp$. Since $Y \neq H$, $H 
\setminus Y$ is not empty. Thus, each $x \in H\setminus Y$ can be written as 
$x+y$ for $x\in Y, y \in Y^\perp$, where $y$ is necessarily not 0, otherwise we 
would have $x+y \in Y$.
 \end{proof}

\end{solution}


\begin{problem}[7][10]
 Let $\func{T}{C\brk{0}{1}}{\R}$ is the linear transformation defined by
 \[T(f) = \Int{f(x)}{x,0,1}.\]
 Suppose that $C\brk{0}{1}$ is equipped with the $\sup$-norm.
\end{problem}

\begin{problem}[7a]
 $T$ is continuous.
\end{problem}

\begin{solution}
 \begin{proof}
  \[\norm{T(f)}_\infty = \norm{\Int{f(x)}{x,0,1}}_\infty \leq \max\pa{\Int{f(x)}{x,0,1}}\norm{f}_\infty\].
  Thus $T$ is bounded and thus continuous.
 \end{proof}

\end{solution}


\begin{problem}[7b]
 Find $\norm{T}$.
\end{problem}

\begin{solution}
 \begin{proof}
  \begin{align*}
   \norm{T} &= \sup \frac{\norm{Tf}}{\norm{f}} \\
   {} &= \sup \frac{\norm{\Int{f(x)}{x,0,1}}_\infty}{\norm{f}_\infty} \\
   {} &= \max\pa{\Int{f(x)}{x,0,1}}
  \end{align*}
 \end{proof}

\end{solution}

\begin{problem}[8][15]
 Let $\ell^2$ be the set of real sequences $x = \pa{x_1,x_2,\dotsc}$ such that $\Sum{\Abs{x_n}^2}{n} < \infty$.
\end{problem}

\begin{problem}[8a]
 Let $T$ be the linear transformation defined by \[T\pa{x_1,x_2,x_3,x_4,\dotsc} = \pa{0,4x_1,x_2,4x_3,x_4,\dotsc}.\] Then $\func{T}{\ell^2}{\ell^2}$ is continuous.
\end{problem}

\begin{solution}
 \begin{proof}
  Let $c \in \ell^\infty$ be defined by $\pa{0,4,1,4,1,\dotsc}$. For every $x 
\in \ell^2$, we have 
  \[\norm{Tx}^2 = \sum_n\Abs{c_i x_i}^2 = (\sup c_i)^2 \sum_n 
\Abs{x_i}^2=16\norm{x}^2.\]
Thus, every $Tx$ is bounded and therefore continuous.
 \end{proof}

\end{solution}


\begin{problem}[8b]
 Find $\norm{T}$.
\end{problem}

\begin{solution}
 Again, we can consider $Tx = cx$ with $c$ defined as above. So $\norm{T} = 
\norm{c}_\infty = 4$.
\end{solution}


\begin{problem}[8c]
 Find $T^2$ and $\norm{T^2}$.
\end{problem}

\begin{solution}
 Since $T$ is defined as \[T\pa{x_1,x_2,x_3,x_4,\dotsc} = \pa{0,4x_1,x_2,4x_3,x_4,\dotsc},\] we can define $T^2$ by \[T^2\pa{x_1,x_2,x_3,x_4,\dotsc} = \pa{0,0,4x_1,4x_2,4x_3,4x_4,\dotsc}.\]
 
 By a similar argument as above, we have $\norm{T^2} = 4$, also.
\end{solution}


\end{document}
